\documentclass[11pt,]{article} % For LaTeX2e

% if you need to pass options to natbib, use, e.g.:
% \PassOptionsToPackage{numbers, compress}{natbib}
% before loading nips_2016
%
% to avoid loading the natbib package, add option nonatbib:
% \usepackage[nonatbib]{nips_2016}

% \usepackage{nips_2016}

% to compile a camera-ready version, add the [final] option, e.g.:
% \usepackage{nips_2016}

\usepackage[utf8]{inputenc} % allow utf-8 input
\usepackage[T1]{fontenc}    % use 8-bit T1 fonts
\usepackage{nicefrac}       % compact symbols for 1/2, etc.
\usepackage{microtype}      % microtypography
\usepackage{amsmath,amssymb,amsthm,amsfonts}
\usepackage{caption,subcaption}
\usepackage{times}
\usepackage{xr}
\usepackage{hyperref}
\usepackage{url}
\usepackage{amssymb}
\usepackage{amsbsy}
\usepackage{amsthm}
\usepackage{amsmath}
\usepackage{paralist}
\usepackage{bm}
\usepackage{booktabs}
\usepackage{rotating}
\usepackage{cases}
\usepackage{mathtools}
\usepackage{graphicx}
\usepackage{algorithm}
\usepackage{algpseudocode}
\usepackage{setspace}
\usepackage{textcomp}
\usepackage{graphicx}
\usepackage{multirow}
\usepackage{color}

\newcommand{\R}{\mathbb{R}}
\newcommand{\I}[1]{\mathbb{I}\left[#1\right]}
\newcommand{\vnorm}[1]{\left|\left|#1\right|\right|}

\title{Probabilistic estimation of Antarctic surface and bed levels using ice-penetrating radar}

% The \author macro works with any number of authors. There are two
% commands used to separate the names and addresses of multiple
% authors: \And and \AND.
%
% Using \And between authors leaves it to LaTeX to determine where to
% break the lines. Using \AND forces a line break at that point. So,
% if LaTeX puts 3 of 4 authors names on the first line, and the last
% on the second line, try using \AND instead of \And before the third
% author name.

\author{
Alyssa D. Jones\footnote{Department of Mathematics, University of Texas at Austin, \texttt{alyssadanijones@gmail.com}} \\
Wesley Tansey\footnote{Department of Computer Science, University of Texas at Austin, \texttt{tansey@cs.utexas.edu} (corresponding author)} \\
Jamin S.~Greenbaum\footnote{Institute for Geophysics, University of Texas at Austin, \texttt{jamin@utexas.edu}} \\
James G.~Scott\footnote{Department of Information, Risk, and Operations Management; Department of Statistics and Data Sciences; University of Texas at Austin, \texttt{james.scott@mccombs.utexas.edu}}
}
\newcommand{\fix}{\marginpar{FIX}}
\newcommand{\new}{\marginpar{NEW}}
\pdfminorversion=4


\begin{document}

\maketitle

\begin{abstract}
Aerial ice-penetrating radargrams are a vital tool for determining ice thickness, presence of subglacial lakes, and many other important features of the Antarctic continent. Currently, human experts must hand-label both the surface and bed lines in each radargram, making the task both laborious and expensive. We present an approach to automating this process via Bayesian statistical inference. We note that both the surface and bed lines create a skew in the underlying signal that fits well to a negative binomial distribution with the human-identified surface and bed points corresponding to the means of the distributions. We model each radargram as a smoothly-changing mixture of surface, bed, and background-noise distributions. In addition to automating the surface and bed detection, our approach also provides uncertainty quantification that is especially valuable for low-signal regions with missing bed lines.
\end{abstract}

\begin{spacing}{1.7}


% !TEX root = main.tex
\section{Introduction}
\label{sec:introduction}
Let $X$ be an $N \times M$ radargram image with $x_{ij} \in \{0, \ldots, 255\}$ being a pixel intensity value corresponding to the radar signal strength for the $i^\text{th}$ location and the $j^\text{th}$ depth. We model $X$ as arising from a smoothly-changing mixture model,

% \begin{equation}
% \label{eqn:model}
\begin{align}
P(X) &= \prod_{i=1}^N P(\mathbf{x}_i | \boldsymbol\theta_i = \{ \mathbf{w}_i, r_i^s, \beta_i^s, r_i^b, \beta_i^b \}) P(\Theta) \\ 
P(\mathbf{x}_i | \boldsymbol\theta_i) &= \prod_{j=1}^M \left[ w_{i}^a\text{Unif}(j) + w_{i}^s\text{NB}(j | r_i^s, \sigma(\beta_i^s)) + w_{i}^b\text{NB}(j | r_i^b, \sigma(\beta_i^b)) \right]^{x_{ij}} \\
P(\Theta) &= \text{GTF}(\Theta | \lambda, \gamma) \, ,
\end{align}
% \end{aligned}
% \end{equation}
where $\mathbf{w}$ are the mixture weights, $(r_i^s, \beta_i^s)$ are the surface component parameters, $(r_i^b, \beta_i^b)$ are the bed component parameters, $\sigma$ is the logistic function, $\text{GTF}$ is the group trend filtering prior distribution with order $\gamma$; $\lambda$ can be included in the hierarchical model or fit via a grid search. The uniform component is used to account for observed constant-offset bias across an entire radargram. We also add the constraint that the mean of the surface component must not be greater than the mean of the surface component,
\begin{equation}
\label{eqn:surface_bed_constraint}
\frac{\sigma(\beta_i^s) r_i^s}{1 - \sigma(\beta_i^s)} \leq \frac{\sigma(\beta_i^b) r_i^b}{1 - \sigma(\beta_i^b)} \,, \qquad i = 1, \ldots, n \, .
\end{equation}

\end{spacing}

\begin{small}
\bibliographystyle{abbrvnat}
\bibliography{main}
\end{small}

\end{document}
